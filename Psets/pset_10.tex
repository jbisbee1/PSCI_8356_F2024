% Options for packages loaded elsewhere
\PassOptionsToPackage{unicode}{hyperref}
\PassOptionsToPackage{hyphens}{url}
%
\documentclass[
]{article}
\usepackage{amsmath,amssymb}
\usepackage{iftex}
\ifPDFTeX
  \usepackage[T1]{fontenc}
  \usepackage[utf8]{inputenc}
  \usepackage{textcomp} % provide euro and other symbols
\else % if luatex or xetex
  \usepackage{unicode-math} % this also loads fontspec
  \defaultfontfeatures{Scale=MatchLowercase}
  \defaultfontfeatures[\rmfamily]{Ligatures=TeX,Scale=1}
\fi
\usepackage{lmodern}
\ifPDFTeX\else
  % xetex/luatex font selection
\fi
% Use upquote if available, for straight quotes in verbatim environments
\IfFileExists{upquote.sty}{\usepackage{upquote}}{}
\IfFileExists{microtype.sty}{% use microtype if available
  \usepackage[]{microtype}
  \UseMicrotypeSet[protrusion]{basicmath} % disable protrusion for tt fonts
}{}
\makeatletter
\@ifundefined{KOMAClassName}{% if non-KOMA class
  \IfFileExists{parskip.sty}{%
    \usepackage{parskip}
  }{% else
    \setlength{\parindent}{0pt}
    \setlength{\parskip}{6pt plus 2pt minus 1pt}}
}{% if KOMA class
  \KOMAoptions{parskip=half}}
\makeatother
\usepackage{xcolor}
\usepackage[margin=1in]{geometry}
\usepackage{color}
\usepackage{fancyvrb}
\newcommand{\VerbBar}{|}
\newcommand{\VERB}{\Verb[commandchars=\\\{\}]}
\DefineVerbatimEnvironment{Highlighting}{Verbatim}{commandchars=\\\{\}}
% Add ',fontsize=\small' for more characters per line
\usepackage{framed}
\definecolor{shadecolor}{RGB}{248,248,248}
\newenvironment{Shaded}{\begin{snugshade}}{\end{snugshade}}
\newcommand{\AlertTok}[1]{\textcolor[rgb]{0.94,0.16,0.16}{#1}}
\newcommand{\AnnotationTok}[1]{\textcolor[rgb]{0.56,0.35,0.01}{\textbf{\textit{#1}}}}
\newcommand{\AttributeTok}[1]{\textcolor[rgb]{0.13,0.29,0.53}{#1}}
\newcommand{\BaseNTok}[1]{\textcolor[rgb]{0.00,0.00,0.81}{#1}}
\newcommand{\BuiltInTok}[1]{#1}
\newcommand{\CharTok}[1]{\textcolor[rgb]{0.31,0.60,0.02}{#1}}
\newcommand{\CommentTok}[1]{\textcolor[rgb]{0.56,0.35,0.01}{\textit{#1}}}
\newcommand{\CommentVarTok}[1]{\textcolor[rgb]{0.56,0.35,0.01}{\textbf{\textit{#1}}}}
\newcommand{\ConstantTok}[1]{\textcolor[rgb]{0.56,0.35,0.01}{#1}}
\newcommand{\ControlFlowTok}[1]{\textcolor[rgb]{0.13,0.29,0.53}{\textbf{#1}}}
\newcommand{\DataTypeTok}[1]{\textcolor[rgb]{0.13,0.29,0.53}{#1}}
\newcommand{\DecValTok}[1]{\textcolor[rgb]{0.00,0.00,0.81}{#1}}
\newcommand{\DocumentationTok}[1]{\textcolor[rgb]{0.56,0.35,0.01}{\textbf{\textit{#1}}}}
\newcommand{\ErrorTok}[1]{\textcolor[rgb]{0.64,0.00,0.00}{\textbf{#1}}}
\newcommand{\ExtensionTok}[1]{#1}
\newcommand{\FloatTok}[1]{\textcolor[rgb]{0.00,0.00,0.81}{#1}}
\newcommand{\FunctionTok}[1]{\textcolor[rgb]{0.13,0.29,0.53}{\textbf{#1}}}
\newcommand{\ImportTok}[1]{#1}
\newcommand{\InformationTok}[1]{\textcolor[rgb]{0.56,0.35,0.01}{\textbf{\textit{#1}}}}
\newcommand{\KeywordTok}[1]{\textcolor[rgb]{0.13,0.29,0.53}{\textbf{#1}}}
\newcommand{\NormalTok}[1]{#1}
\newcommand{\OperatorTok}[1]{\textcolor[rgb]{0.81,0.36,0.00}{\textbf{#1}}}
\newcommand{\OtherTok}[1]{\textcolor[rgb]{0.56,0.35,0.01}{#1}}
\newcommand{\PreprocessorTok}[1]{\textcolor[rgb]{0.56,0.35,0.01}{\textit{#1}}}
\newcommand{\RegionMarkerTok}[1]{#1}
\newcommand{\SpecialCharTok}[1]{\textcolor[rgb]{0.81,0.36,0.00}{\textbf{#1}}}
\newcommand{\SpecialStringTok}[1]{\textcolor[rgb]{0.31,0.60,0.02}{#1}}
\newcommand{\StringTok}[1]{\textcolor[rgb]{0.31,0.60,0.02}{#1}}
\newcommand{\VariableTok}[1]{\textcolor[rgb]{0.00,0.00,0.00}{#1}}
\newcommand{\VerbatimStringTok}[1]{\textcolor[rgb]{0.31,0.60,0.02}{#1}}
\newcommand{\WarningTok}[1]{\textcolor[rgb]{0.56,0.35,0.01}{\textbf{\textit{#1}}}}
\usepackage{graphicx}
\makeatletter
\def\maxwidth{\ifdim\Gin@nat@width>\linewidth\linewidth\else\Gin@nat@width\fi}
\def\maxheight{\ifdim\Gin@nat@height>\textheight\textheight\else\Gin@nat@height\fi}
\makeatother
% Scale images if necessary, so that they will not overflow the page
% margins by default, and it is still possible to overwrite the defaults
% using explicit options in \includegraphics[width, height, ...]{}
\setkeys{Gin}{width=\maxwidth,height=\maxheight,keepaspectratio}
% Set default figure placement to htbp
\makeatletter
\def\fps@figure{htbp}
\makeatother
\setlength{\emergencystretch}{3em} % prevent overfull lines
\providecommand{\tightlist}{%
  \setlength{\itemsep}{0pt}\setlength{\parskip}{0pt}}
\setcounter{secnumdepth}{-\maxdimen} % remove section numbering
\ifLuaTeX
  \usepackage{selnolig}  % disable illegal ligatures
\fi
\usepackage{bookmark}
\IfFileExists{xurl.sty}{\usepackage{xurl}}{} % add URL line breaks if available
\urlstyle{same}
\hypersetup{
  pdftitle={Problem Set 6},
  pdfauthor={{[}YOUR NAME{]}},
  hidelinks,
  pdfcreator={LaTeX via pandoc}}

\title{Problem Set 6}
\usepackage{etoolbox}
\makeatletter
\providecommand{\subtitle}[1]{% add subtitle to \maketitle
  \apptocmd{\@title}{\par {\large #1 \par}}{}{}
}
\makeatother
\subtitle{Quantitative Political Science}
\author{{[}YOUR NAME{]}}
\date{Due Date: 2024-11-18}

\begin{document}
\maketitle

\section{Instructions}\label{instructions}

For this assignment, you are to use the \texttt{counties.dta} dataset
found on our class
\href{https://github.com/jbisbee1/PSCI_8356_F2024/raw/refs/heads/main/Data/counties.dta}{github}.
It comes from the City and County Data Book produced by the U.S. Census
Bureau.

\begin{Shaded}
\begin{Highlighting}[]
\FunctionTok{require}\NormalTok{(tidyverse)}
\FunctionTok{require}\NormalTok{(haven)}

\NormalTok{dat }\OtherTok{\textless{}{-}} \FunctionTok{read\_dta}\NormalTok{(}\StringTok{\textquotesingle{}https://github.com/jbisbee1/PSCI\_8356\_F2024/raw/refs/heads/main/Data/counties.dta\textquotesingle{}}\NormalTok{)}
\end{Highlighting}
\end{Shaded}

\paragraph{\texorpdfstring{1: Analyze the following questions using
\texttt{R}, but answer them with a few sentences in plain
English.}{1: Analyze the following questions using R, but answer them with a few sentences in plain English.}}\label{analyze-the-following-questions-using-r-but-answer-them-with-a-few-sentences-in-plain-english.}

\begin{enumerate}
\def\labelenumi{\alph{enumi}.}
\item
  Controlling for household income (\texttt{hhincome}), is the
  percentage of a county's residents who have college degrees
  (\texttt{college}) associated with crime rates (\texttt{crimerate})?
  How so?
\item
  Construct a plot displaying the relationship between
  \(\widehat{\textit{crimerate}}\) and \texttt{college}. To do this,
  write a loop that generates the value of
  \(\widehat{\textit{crimerate}}\) when \texttt{college} is equal to
  \(5, 7,\dots,25\). (In doing so, you may find it helpful to use the
  coefficients.) Then plot these connected points.
\item
  You are interested in how well this model applies to counties in New
  York State. Construct a plot showing at a glance that Tompkins County,
  has a crime rate much lower than predicted by a regression of
  \texttt{crimerate} on \texttt{college} among counties in New York
  State.
\item
  Look up Tompkins County on the Internet. Why might it be an outlier?
  Does this suggest an additional control variable?
\end{enumerate}

\paragraph{2: Is unemployment higher in counties located in the South
than those located outside the
South?}\label{is-unemployment-higher-in-counties-located-in-the-south-than-those-located-outside-the-south}

\begin{enumerate}
\def\labelenumi{\alph{enumi}.}
\item
  Answer this question first with a \(t\)-test to compare group means.
\item
  Now answer this question by running a bivariate specification
  regressing \texttt{unemprate} on the indicator variable
  \texttt{south}.
\item
  Identify two important similarities found between the two analyses.
\item
  We require a lot more assumptions to perform the second analysis than
  the first. In a few sentences, reflect upon why they are unnecessary
  to answer the question posed above.
\end{enumerate}

\paragraph{3: Is the association between poverty and crime by county
stronger in the South or outside the
South?}\label{is-the-association-between-poverty-and-crime-by-county-stronger-in-the-south-or-outside-the-south}

\begin{enumerate}
\def\labelenumi{\alph{enumi}.}
\item
  Answer this question first by running two separate regressions.
\item
  Now answer this question using an interacted specification.
\item
  Identify two important similarities found between the estimated
  generated by the two analyses.
\item
  Construct a plot with two lines: one showing
  \(\widehat{\textit{crimerate}}\) by \texttt{povrate} among counties in
  the South, and the other showing \(\widehat{\textit{crimerate}}\) by
  \texttt{povrate} among non-Southern counties.
\end{enumerate}

\paragraph{4: Perform the necessary analyses to assess the accuracy of
the following statements. In a few sentences, discuss each of your
findings.}\label{perform-the-necessary-analyses-to-assess-the-accuracy-of-the-following-statements.-in-a-few-sentences-discuss-each-of-your-findings.}

\begin{enumerate}
\def\labelenumi{\alph{enumi}.}
\item
  The bivariate relationship between poverty and crime is weaker where
  local government spending (\texttt{locspendcap}) is higher.
\item
  The bivariate relationship between unemployment and crime is stronger
  in densely populated counties (\texttt{density}).
\item
  Federal spending per capita (\texttt{fedspendcap}) plays a greater
  role in reducing poverty than local spending per capita
  (\texttt{locspendcap}), controlling for variables that may confound
  these relationships.
\end{enumerate}

\paragraph{\texorpdfstring{5: Pretend to \textbf{not} be a dispassionate
scholar but instead are someone with an agenda who wants to show that
crime is caused by poverty. To do so, you are hell-bent on finding the
combination of regressors (including \texttt{povrate}) that results in a
very large coefficient on \texttt{povrate} when \texttt{crimerate} is
regressed upon these
regressors.}{5: Pretend to not be a dispassionate scholar but instead are someone with an agenda who wants to show that crime is caused by poverty. To do so, you are hell-bent on finding the combination of regressors (including povrate) that results in a very large coefficient on povrate when crimerate is regressed upon these regressors.}}\label{pretend-to-not-be-a-dispassionate-scholar-but-instead-are-someone-with-an-agenda-who-wants-to-show-that-crime-is-caused-by-poverty.-to-do-so-you-are-hell-bent-on-finding-the-combination-of-regressors-including-povrate-that-results-in-a-very-large-coefficient-on-povrate-when-crimerate-is-regressed-upon-these-regressors.}

\begin{enumerate}
\def\labelenumi{\alph{enumi}.}
\item
  Choose 90 percent of your observations at random and set them aside.
  Do not include these observations in the analysis.
\item
  Find the combination of variables that yields the highest possible
  coefficient on \texttt{povrate} as a predictor of \texttt{crimerate}
  that you can obtain.
\item
  Once you've arrived at a specification, note the results. Now, take
  the rest of your dataset out of ``cold storage.'' How well does your
  model do with these other data? What is the lesson learned?
\end{enumerate}

\end{document}
